
\documentclass{article}
\usepackage[margin=2.4cm,a4paper]{geometry}
\usepackage[T1]{fontenc}
\usepackage{longtable}
\usepackage[colorlinks]{hyperref}

\title{Alloy: A Quick Reference}
\author{Martin Monperrus\\ with contributions of Marc Frappier}



\newlength\tw
\setlength{\tw}{7.5cm}
\begin{document}

\maketitle

\noindent This document presents the syntax and intuitive meaning of the main concepts of the Alloy modeling language. It somehow focuses on the "navigation expression style " of Alloy (see Jakson's "Software Abstractions", p.34) and refers to classical object-oriented knowledge. It is inspired from Mohammad Mousavi's "Z and Alloy Notations: A Quick Reference".
\medskip

\noindent Contributions are welcome as pull requests on:\\ \url{https://github.com/monperrus/alloy-quick-reference}

\section{Basics}

\begin{longtable}{|p{\tw}|p{\tw}|}
\hline
\emph{Notation} & \emph{Intuitive Meaning} \\
\hline
\texttt{sig Book \{ \}} & Class/type definition \\
\hline
\texttt{sig Dictionary extends Book \{ ... \}} & Inheritance \\
\hline
\texttt{sig Book \{ author: Author \}} & Field, reference, cannot be null \\
\hline
\texttt{b.author} & Field access\\
\hline
\texttt{pred predicate\_name \{ ... \}} & Predicate (returns true or false), property definition \\
\hline
\texttt{pred has\_author[b : Book] \{  \}} & Parametrized property definition \\
\hline
\texttt{fact \{ ... \}} & Predicate that always holds, a formula which must be satisfied by all instances of an Alloy model\\
\hline
\texttt{run \{ ... \}} & Find instances \\
\hline
\texttt{run \{ ... \} for 3 but 1 Author} & Find instances with constraints on # of instances\footnote{except for intergers (Int) where n is the bitwidth}\\
\hline
\texttt{pred foo[b:Book] \{ .... \}\newline
run foo for 3 but 1 Author} & Find instances satisfying predicate "foo" \\
\hline
\texttt{assert GaGb \{ good\_author implies  good\_book \}\newline
check GaGb} & Find counter-examples violating the assertion \\
\hline
\texttt{check \{ good\_author implies  good\_book \} } & Anonymous assertions \\
\hline
\texttt{fun books(a: Author) : set Book {...}} & Function definition (side-effect free), returns an expression of some type (here \texttt{set Book}) \\
\hline
\texttt{let x: E | \ldots} & Identifier definition\\
\hline
\end{longtable}


\section{First Order Logic}

\begin{longtable}{|p{\tw}|p{\tw}|}
\hline
\emph{Notation} & \emph{Intuitive Meaning} \\
\hline
\texttt{p and q, p \&\& q} & Conjunction, logical and \\
\hline
\texttt{p or q, p || q} & Disjunction, logical or\\
\hline
\texttt{p implies q, p => q} & Implication \\
\hline
\texttt{p iff q, p <=> q} & Equivalence \\
\hline
\texttt{not p, !p} & Negation \\
\hline
\texttt{all b:Book | has\_author[b]} & Universal quantification, for all \\
\hline
\texttt{some b:B | has\_author[b]} & Existential quantification, at least one \\
\hline
\texttt{some disjoint b1, b2 : B | \ldots} & Existential quantification, forcing removal of case $b1=b2$\\
\hline
\end{longtable}

\section{Set Theory}

\begin{longtable}{|p{\tw}|p{\tw}|}
\hline
\emph{Notation} & \emph{Intuitive Meaning} \\
\hline
\texttt{Book} & All instances of Book \\
\hline
\texttt{sig Dictionary extends Book \{ ... \}} & Set Dictionary is a subset of set Book, all extension signatures (subclasses) are disjoint. \\
\hline
\texttt{univ} & All instances of all types (the universe) \\
\hline
\texttt{none} & Empty set \{\} \\
\hline
\texttt{no x} & Set $x$ is empty\\
\hline
\texttt{\#Book} & Cardinality \\
\hline
\texttt{a in b} & Subset or equal\\
\hline
\texttt{a = b} & Set equality\\
\hline
\texttt{some x} & Set not empty, $|x|\geq1$\\
\hline
\texttt{one x} & $|x|=1$ \\
\hline
\texttt{lone x} & $|x|\leq1$\\
\hline
\texttt{x \& y} & Intersection\\
\hline
\texttt{x + y } & Union\\
\hline
\texttt{x - y} & Difference\\
\hline
\end{longtable}

\section{Objects}

\begin{longtable}{|p{\tw}|p{\tw}|}
\hline
\emph{Notation} & \emph{Intuitive Meaning} \\
\hline
\texttt{abstract sig Book \{\}} & Abstract class, cannot be instantiated \\
\hline
\texttt{one sig Bible extends Book \{...\}} & Singleton, $|Bible| = 1$, Bible subset of Book \\
\hline
\texttt{enum Vegetable \{Potato, Carrot, Tomato\}} & Enumeration \\
\hline
\texttt{sig Book \{ author: Author \}} & Field, reference, cannot be null \\
\hline
\texttt{b.author} & Field access\\
\hline
\texttt{b.isGood} & Method call without parameters\newline (with \texttt{pred  isGood[x:Book]}) \\
\hline
\texttt{b.isBetterThan[b2]} & Method call with parameters\newline (with \texttt{pred  isGood[x,y: Book]}) \\
\hline
\texttt{sig Book \{ author: Author \}} & Multiplicity $[1\ldots1]$ \\
\hline
\texttt{sig Book \{ author: set Author \}} & Multiplicity $[0\ldots\star]$ \\
\hline
\texttt{sig Book \{ author: lone Author \}} & Multiplicity $[0\ldots\star]$ \\
\hline
\texttt{sig Library \{ books: Author -> Book \}} & Dictionary, hashtable, cartesian product\\
\hline
\texttt{myLibrary.books[a]} & Dictionary member access\\
\hline
\texttt{dom[Library.books]} & Keys of the dictionary (with "open util/relation")\\
\hline
\texttt{ran[Library.books]} & Values of the relation (with "open util/relation") \\
\hline
\end{longtable}

\section{Relations}

\begin{longtable}{|p{\tw}|p{\tw}|}
\hline
\emph{Notation} & \emph{Intuitive Meaning} \\
\hline
\texttt{a->b} & Cartesian product $a \times b$ \\
\hline
\texttt{a.b} & Relational product\\
\hline
\texttt{a<:b} & Domain restriction of relation b by set a \\
\hline
\texttt{a:>b} & Range restriction of relation b by set a \\
\hline
\texttt{joe.\^{}friends} & Transitive closure\\
\hline
%\texttt{*friends} & Reflexive transitive closure\\
%\hline
\texttt{\~{}x} & Inverse \\
\hline
\texttt{T-> one U} &  Total function from T to U\\
\hline
\texttt{T-> lone U} & Partial function from T to U\\
\hline
\texttt{T one -> one U} & Bijection from T to U\\
\hline
\end{longtable}

\section{Integer Arithmetic}

When using integer, add \texttt{open util/integer} in preamble and always specify the integer bitwidth (\texttt{run \{\ldots\} for 5 Int}). There is no real or floating-point arithmetic in Alloy.

\begin{longtable}{|p{\tw}|p{\tw}|}
\hline
\emph{Notation} & \emph{Intuitive Meaning} \\
\hline
\texttt{plus[a,b]} & Addition\\
\hline
\texttt{minus[a,b]} & Substraction\\
\hline
\texttt{mul[a,b]} & Multiplication\\
\hline
\texttt{div[a,b]} & Division\\
\hline
\texttt{rem[a,b]} & Remainder of a divided by b\\
\hline
\texttt{sum[a]} & Returns the sum of the integers of set $a$\\
\hline
\texttt{a < b, a = b, a > b, a =< b, a >= b} & Integer comparison\\
\hline
\end{longtable}

\section{Ordering}

\begin{longtable}{|p{\tw}|p{\tw}|}
\hline
\emph{Notation} & \emph{Intuitive Meaning} \\
\hline
\texttt{open util/ordering[State] as states} & Declares a total order on State\\
\hline
\texttt{states/first} & First element\\
\hline
\texttt{states/last} & Last element\\
\hline
\texttt{states/next[s]} & Next element\\
\hline
\texttt{states/prev[s]} & Previous element\\
\hline
\texttt{states/nexts[s]} & All elements after s\\
\hline
\texttt{states/prev[s]} & All elements before s\\
\hline
\end{longtable}

\section{Sequences / Lists}}
%% see \url{http://alloy.mit.edu/alloy/documentation/quickguide/seq.html}
\begin{longtable}{|p{\tw}|p{\tw}|}
\hline
\emph{Notation} & \emph{Intuitive Meaning} \\
\hline
\texttt{s : seq A} & Ordered and indexed sequence of elements\\
\hline
\texttt{s.first} & Head of the list\\
\hline
\texttt{s.rest} & Tail of the list\\
\hline
\texttt{s.elems} & All elements as an unordered set\\
\hline
\texttt{s.idxOf [x]} & Returns the first index where x appears in s, if x does not appear in s, it returns the empty set.\\
\hline
\texttt{s.insert[i, x]} &  Returns a new list where x is inserted at index i\\
\hline
\end{longtable}

\newpage
\section{Precedence}

In increasing order; operators on the same line have the same priority.

\begin{minipage}{\tw}
Expressions (operands are not booleans)
\begin{enumerate}
\itemsep=0em
\item \~{} , \^{}  , *
\item .
\item ~[]
\item <: , :>
\item ->
\item &
\item ++
\item #
\item +, -
\item no , some , lone , one , set
\item ! , not
\item in , = , < , > , = , =< , =>
\end{enumerate}
\end{minipage}
\begin{minipage}{\tw}
Logical expressions (operands are booleans) 
\begin{enumerate}
\itemsep=0em
\item ! ,, not
\item \&\& , and
\item => , implies , else
\item <=> , iff
\item || , or
\item let , no , some , lone , one 
\end{enumerate}
\\~
\\~
\\~
\\~
\\~
\\~
\\~
\end{minipage}

\end{document}